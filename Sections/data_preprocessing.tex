\section{Data Preprocessing}\label{data_preprocessing}

This section encompasses various preparatory procedures aimed at optimizing the data for subsequent analysis. 

\subsection{Dataset}\label{dataset_given}

The dataset encompasses a singular class and fifteen distinctive features. Among these, the feature designated as \Gls{cir} notably consists of 1016 subfeatures. The features are as follows:

\begin{itemize}
    \item \textbf{Range:} reflects the distance from the transmitter to the receiver.
    \item \textbf{FP\_IDX:} represents the index of the first path.
    \item \textbf{FP\_AMP1, FP\_AMP2, and FP\_AMP3:} indicate the amplitude of the first path signal, providing data for the direct signal path.
    \item \textbf{STDEV\_NOISE:} denotes the standard deviation of noise, a measure of noise variability.
    \item \textbf{CIR\_PWR:} measures the total power found within the \acrshort{cir}, offering a summary of received signal strength.
    \item \textbf{MAX\_NOISE and RXPACC:} serve as metrics for the highest noise level and the accuracy of the receiver's signal processing, respectively.
    \item \textbf{CH, FRAME\_LEN, PREAM\_LEN, BITRATE, and PRFR:} are features associated with the communication protocol parameters. They include the channel number, frame length, preamble length, bitrate, and pulse repetition frequency.
    \item \textbf{CIRO-CIR1015:} contains sequential or time-series data points, an extensive collection that forms part of the \acrshort{cir} data.
\end{itemize}

% \begin{description}[style=nextline]
%     \item[Feature Extraction:] Identification and extraction of pertinent features from the dataset.
%     \item[Feature Derivation:] Derivation of new features from existing ones to enhance predictive power.
%     \item[Feature Reduction:] Reduction of feature dimensionality to mitigate computational complexity.
%     \item[De-Noise Functions:] Application of functions to remove noise and enhance data quality.
%     \item[Justifications of Processes:] Explanations substantiating the rationale behind each preprocessing step.
% \end{description}

\subsection{Feature Extraction}\label{feature_extraction}

No Features were extracted during preprocessing stage due to nature of the models used to model the data.


\subsection{Feature Derivation}\label{feature_derivation}

This section delineates the derivation of essential features from the Decawave Ultra-Wideband (\acrshort{uwb}) Module datasheet, crucial for subsequent analysis. These derived features were used in the data mining phase to train the machine learning models. Most of the features were derived from the formula provided in \cite{DecawaveDW1000}.

\subsubsection{First Path Power Level}\label{first_path_power_level}

The First Path Power Level represents the strength of the primary signal path in the received Channel Impulse Response. It is calculated using the following equation:

\begin{equation}
  \text{First Path Power Level} = 10 \cdot \log_{10} \left( \frac{F_1^2 + F_2^2 + F_3^2}{N^2} \right) - A
\end{equation}

where:

\begin{itemize}
  \item $F_i$ (Current variable name: $FP\_AMPi$): Amplitude of the $i^{th}$ peak in the \acrshort{cir} ($i = 1, 2, 3$)
  \item $N$ (Current variable name: $RXPACC$): Receiver path accumulation length
  \item $A$ (Current variable name: $PRFR$):Pulse Repetition Frequncy Rate (typically 64)
\end{itemize}

\subsubsection{RX Level}\label{rx_level}

The RX Level metric assesses the received signal strength relative to the receiver sensitivity threshold. Its computation is described by the following formula:

\begin{equation}
  \text{RX\_Level} =
  \begin{cases}
    \text{median}(RX\_Values) & \text{if } C = 0 \\
    10 \cdot \log_{10} \left( \frac{C \cdot 2^{17}}{N^2} \right) - A & \text{otherwise}
  \end{cases}
\end{equation}

where:

\begin{itemize}
  \item $C$ (Current variable name: $CIR\_PWR$): Power of the received Channel Impulse Response
  \item $N$ (Current variable name: $RXPACC$): Receiver path accumulation length
  \item $PRFR$ (Current variable name unchanged): Pulse Repetition Frequency Rate in MHz
\end{itemize}

In scenarios where the condition $C = 0$ holds, the RX Level is determined by taking the median value of all RX values. This approach ensures robustness against outliers and prevents the generation of undefined results.


\subsubsection{Signal-to-Noise Ratio (SNR)}\label{snr}

The Signal-to-Noise Ratio (SNR) is a metric used to quantify the quality of the received signal relative to the noise level. It is computed as:

\begin{equation}
  \text{SNR} = \frac{P_{cir}}{\sigma_n}
\end{equation}

where:

\begin{itemize}
  \item $P_{cir}$ (Current variable name: $CIR\_PWR$): Power of the received Channel Impulse Response
  \item $\sigma_n$ (Current variable name $STDEV\_NOISE$): Standard deviation of the noise
\end{itemize}

The SNR provides valuable insight into the reliability of the received signal amidst background noise. A visual representation of the SNR calculated is shown at Figure \ref{fig:snr}

\subsection{Feature Reduction}\label{feature_reduction}
During the dataset refinement phase, we removed attributes that displayed uniform values for all records. This entailed discarding Channel Information (CH), which was invariably recorded as 2, BITRATE, which consistently maintained a value of 110, Pulse Repetition Frequency Rate (PRFR), perpetually set at 64, and Preamble Length (PREAM\_LEN), uniformly established at 1024.

% Double check ALL formula, maybe cite or something idk
\subsection{De-Noise Functions}\label{de_noise_Functions}

In the paper by \cite{jiang_uwb_2020}, the authors proposed a novel denoising approach for UWB signals. They employed the Lucy-Richardson algorithm, a well-known deconvolution algorithm, to remove noise and enhance the quality of the received signal. The algorithm operates by iteratively estimating the original signal from the observed signal and the point spread function (PSF), which characterizes the blurring effect on the signal due to noise and other factors. The authors demonstrated the effectiveness of the Lucy-Richardson algorithm in denoising UWB signals and improving the accuracy of localization systems. The algorithm was shown to effectively remove noise and enhance the quality of the received signal, leading to improved localization accuracy and reliability.

However, they also suggested that the Wavelet Transformation approach is used more commonly in practice due to its simplicity and effectiveness. The Wavelet Transformation approach involves decomposing the signal into different frequency bands using wavelet functions, which allows for the removal of noise and the extraction of important features from the signal. The authors demonstrated the effectiveness of the Wavelet Transformation approach in denoising UWB signals and improving the accuracy of localization systems. The approach was shown to effectively remove noise and enhance the quality of the received signal, leading to improved localization accuracy and reliability.

In this section, we will explore the implementation of the Wavelet Transformation approach to denoise the UWB signals and enhance the quality of the received signal. We will also evaluate the effectiveness of the approach in improving the accuracy of the localization system and discuss its potential advantages and limitations.

\subsubsection{Wavelet Transformation}\label{wavelet_transformation}

The Wavelet Transformation approach involves decomposing signals into different frequency bands using wavelet functions, facilitating noise removal and feature extraction. Widely utilized in signal processing and data analysis, this method is particularly effective for denoising Ultra-Wideband (UWB) signals, enhancing localization accuracy and reliability.

\textbf{Equation for Wavelet Transform:}

The Discrete Wavelet Transform (DWT) decomposes the signal $x(t)$ into wavelet coefficients $W_f(a, \tau)$ at various scales $a$ and translations $\tau$:

\begin{equation}
W_f(a, \tau) = \int_{-\infty}^{\infty} x(t) \psi_{a, \tau}(t) dt
\end{equation}

where:

\begin{itemize}
  \item $W_f(a, \tau)$ represents wavelet coefficients
  \item $x(t)$ is the original signal
  \item $\psi_{a, \tau}(t)$ is the wavelet function (mother wavelet)
  \item $a$ is the scale parameter (controlling wavelet's width)
  \item $\tau$ is the translation parameter (controlling wavelet's position)
\end{itemize}

The denoising process involves applying a thresholding function to the wavelet coefficients, retaining significant signal components while removing noise.

Utilizing the Daubechies wavelet, LOS and NLOS signals are transformed, enhancing signal quality by noise removal. The denoised signals serve as input for training and testing machine learning models.

The Wavelet Transformation approach is instrumental in denoising UWB signals and improving localization accuracy. Its simplicity and effectiveness make it a valuable tool in enhancing signal quality and subsequently improving the performance of machine learning models.

\subsubsection{Lucy-Richardson Algorithm}

Given observed signal $I$ (cir\_values) and point spread function (PSF) $h$ (psf), the Richardson-Lucy algorithm aims to estimate the original signal $J$ through iterative refinement. The algorithm iteratively updates the estimate $J^{(k)}$ according to the following formula:

\begin{equation}
J^{(k+1)} = J^{(k)} \otimes \left( \frac{I}{(J^{(k)} \otimes h)} \right) \otimes h^T
\end{equation}

Where:
\begin{itemize}
    \item $J^{(k)}$ is the estimate of the original signal at iteration $k$.
    \item $\otimes$ denotes the convolution operation.
    \item $h^T$ represents the transpose of the PSF. 
\end{itemize}


The algorithm terminates after a specified number of iterations or upon reaching convergence criteria.

In the context of Python programming, this algorithm can be implemented using the following function:

\begin{verbatim}
def deconvolve_cir(cir_values, psf=None, iterations=50):
    # If no point spread function is provided, create a simple one
    if psf is None:
        psf = np.ones((5,)) / 5

    # Perform Richardson-Lucy deconvolution
    deconvolved_cir = restoration.richardson_lucy(
        cir_values, 
        psf, 
        num_iter=iterations
    )

    return deconvolved_cir
\end{verbatim}

This function takes shifted \acrshort{cir} values as input and performs Richardson-Lucy deconvolution. If no PSF is provided, a simple one is created. The number of iterations can also be specified. The result is the deconvolved \acrshort{cir}.

A visualization depicting the frequency graph of the wavelet-transformed \acrshort{los} and \acrshort{nlos} signals is presented in Section \ref{frequency_graph_lr}. Notably, Figure \ref{fig:frequency_graph_lr} exhibits an almost linear trend across the graph, suggesting a potential misapplication of the underlying function. Conversely, Figure \ref{fig:frequency_graph_lr_scaled} portrays a cleaned and scaled rendition of the frequency graph. Despite its improved clarity, this visualization was omitted from the Data Mining phase due to its notably low variance, as evidenced by both figures.

\subsubsection{\Gls{Discrete Fourier Transform} in Machine Learning for Signal Processing}

The \acrshort{dft} serves as a pivotal tool in machine learning projects aimed at signal processing, particularly in predicting the presence of \acrshort{los} or \acrshort{nlos} conditions from a dataset. In this context, the \acrshort{dft} aids in analyzing the frequency characteristics of signals transmitted over wireless channels, offering crucial insights into channel behavior. Inspired by \cite{8682194}, we wanted to perform \acrshort{dft}, followed by training a \Gls{Convolutional Neural Network} and \Gls{Multilayer Perceptron} model in hopes of a better accuracy, after performing training on the wavelet-transformed \acrshort{cir} data.

\begin{itemize}
    \item \textbf{Data Extraction:} Initially, the dataset containing \acrshort{cir} values is processed. \acrshort{cir} values, representing the response of the wireless channel to an impulse signal, are extracted from the dataset. These values serve as the input for subsequent DFT computations.
    
    \item \textbf{\acrshort{dft} Computation:} The \acrshort{dft} is computed on the extracted \acrshort{cir} sequence utilizing the \texttt{numpy.fft.fft} (Fast Fourier Transform) function. Specifically, the \acrshort{dft} calculation is executed along the second axis (columns) of the data array, ensuring that frequency analysis is performed accurately.
    
    \item \textbf{Magnitude Spectrum Calculation:} As the \acrshort{dft} output comprises complex numbers, the next step involves computing the absolute values to derive the magnitude spectrum. This spectrum encapsulates the strengths of various frequency components present within the signal, facilitating a comprehensive understanding of its frequency characteristics.
    
    \item \textbf{Output Generation:} Finally, the function yields the absolute values (magnitude spectrum) of the \acrshort{dft}, which encapsulate vital information regarding the frequency composition of the input signal.
\end{itemize}

The mathematical representation of the \acrshort{dft} underscores its significance in signal analysis:

\begin{equation}
X[k] = \sum_{n=0}^{N-1} x[n] \cdot e^{-i(2\pi/N)kn}
\end{equation}

where:
\begin{itemize}
    \item $X[k]$: Represents the $k$-th element of the DFT output sequence.
    \item $x[n]$: Denotes the $n$-th element of the input \acrshort{cir} sequence.
    \item $N$: Signifies the total number of elements (length) in the \acrshort{cir} sequence.
    \item $i$: Represents the imaginary unit.
\end{itemize}

By leveraging the DFT within machine learning frameworks, we can discern crucial patterns in wireless channel behavior, aiding in the prediction of NLOS or LOS conditions. These insights are invaluable for optimizing communication systems and mitigating the effects of multipath propagation, where signals traverse multiple paths, introducing delays and potential signal degradation.

A visualization of the \acrshort{cir} values after DFT is shown in Section \ref{dft_visual}.

\subsection{Summary and Justifications}

In the model training process, certain features were selected while others were omitted based on their relevance and potential impact on model performance. Specifically, features such as RX Level, First Path Power Level, and Signal to Noise Ratio were deemed significant and thus retained for inclusion in the model training dataset. Conversely, features such as Channel ('CH') and Bitrate ('BITRATE') were excluded due to possessing only a single unique value, rendering them incapable of providing meaningful information for model learning and inference.

Furthermore, the decision was made to replace the \acrshort{cir} columns ('CIR0 - CIR1015') with their Wavelet-Transformed counterparts, as detailed in Section \ref{wavelet_transformation}. This transformation was chosen to enhance the model's ability to discern salient patterns within the \acrshort{cir} signals, thereby improving classification accuracy.

Regarding the choice not to pre-reduce features before inputting them into the \gls{cnn} or \gls{mlp}, it was reasoned that manually extracting features might overlook crucial information pertinent to the classification task. \acrshort{cnn}s are adept at autonomously learning discriminative features directly from raw data during the training phase, a capability that often leads to superior model performance. By allowing the \acrshort{cnn} to extract features from the UWB \acrshort{cir} signals organically, the model can effectively discern the most relevant information for accurate classification, thereby mitigating the risk of feature reduction potentially discarding crucial discriminative features.

