\section{Conclusion}\label{conclusion}

This work comprehensively investigated the impact of data preprocessing and machine learning model selection on signal data analysis. The investigation explored various preprocessing techniques, including denoising via wavelet transform and lucy-richardson deconvolution, to enhance data quality. Subsequently, the Discrete Fourier Transform (DFT) was employed to extract informative frequency domain features for model training. These meticulous preprocessing steps established a robust foundation for subsequent machine learning endeavors.

Following a rigorous process of research and experimentation, a Convolutional Neural Network (CNN) model emerged as demonstrably superior to its Multilayer Perceptron (MLP) counterpart. The inherent strength of CNNs in feature extraction, particularly emphasized by the convolutional layers, proved instrumental in achieving optimal performance for the specific requirements of this task. Further, fine-tuning the CNN architecture yielded significantly better results compared to the MLP, solidifying the suitability of CNNs for signal data analysis tasks.

The visualization stage revealed that both models exhibited commendable performance on both denoised and DFT-transformed datasets. This observation underscores the importance of effective data preprocessing in achieving model robustness and generalizability, even on unseen data. In-depth examination of model weights, biases, confusion matrices, and Receiver Operating Characteristic (ROC) curves provided further confirmation of these findings.

In conclusion, this work highlights the critical interplay between data preprocessing, model selection, and visualization techniques in ensuring success in data analytics applications. The achieved results not only validate the effectiveness of the employed methodologies but also illuminate the potential of advanced neural network architectures, such as CNNs, for addressing intricate signal data analysis challenges. This work further emphasizes the importance of domain knowledge in guiding these stages, underscoring the need for deep understanding in both areas for successful machine learning projects.


\section{Future Work}\label{Future Work}
In subsequent endeavors, enhancing our comprehension of the domain is imperative. Careful calibration of model parameters and the judicious selection of data are essential to strike an optimal equilibrium between model accuracy and training duration. Moving forward, it would be beneficial to explore advanced techniques to refine the model's performance further. This includes experimenting with alternative algorithms, adopting more sophisticated preprocessing methods, or integrating additional data sources to enrich the training process. Furthermore, an assessment of the computational efficiency and scalability of the model should be conducted to ensure practical applicability in real-world scenarios. By systematically addressing these areas, we aim to advance the model's capabilities and applicability.